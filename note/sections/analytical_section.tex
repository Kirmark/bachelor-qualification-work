\chapter{Аналитический раздел}
\todo[inline]{25 – 30 страниц}

%
\section{Постановка задачи}
\textbf{Целью} данной работы является разработка метода тематического моделирования для новостей на русском языке.

Для достижения этой цели необходимо выполнить следующие основные \textbf{задачи}:

\begin{itemize}
    \item // Анализ существующих решений и выбор базового алгоритма тематического моделирования для классификация/категоризация новостей на русском языке
    \item Разработка программного продукта для сбора новостей на русском языке и подготовки данных для последующего анализа
    \item Подбор методов улучшения алгоритма и значений их параметров
    \item Обучение модели
    \item // проведение эксперимента
\end{itemize}

%
\section{Анализ предметной области}

\todo[inline]{проводится анализ предметной области}
\todo[inline]{выделяется основной объект исследования}

Классы задач, для которых используется тематическое моделирование разбивают на 2 типа: \textbf{Автоматический анализ текста} и \textbf{систематизация больших объемов информации}.

В задачах автоматического анализ а текста обычно выделяют следующие направления:

\begin{itemize}
    \item Классификация и категоризация документов
    \item Автоматическое аннотирование документов
    \item Автоматическая суммаризация коллекций
    \item Тематическая сегментация документов
\end{itemize}

В задачах систематизации больших объемов информации обычно выделяют следующие направления:

\begin{itemize}
    \item Семантический (разведочный) поиск информации
    \item Визуализация тематической структуры коллекции
    \item Анализ динамики развития тем                      
    \item Тематический мониторинг новых поступлений
    \item Рекомендация документов пользователям
\end{itemize}

%
\section{Существующие методы}
\todo[inline]{обзор существующих путей/методов/решений и алгоритмов решения}
\todo[inline]{Классификация и кластеризация документов, VSM (Vector Space Model)}
\todo[inline]{LSA - Латентно-семантическое индексирование, SVD - Singular Value Decomposition}
\todo[inline]{? Графические модели}
\todo[inline]{PLSA - Probabilistic latent semantic analysis }
\todo[inline]{LDA - Latent Dirichlet allocation - латентное размещение Дирихле - специальный регуляризатор для Баеса}
\todo[inline]{? рLDA}
\todo[inline]{JPM - Join Probabilistic Model, AHMM - Aspect Hidden Markov Model, ATM - Autor-Topic Model, CTM - Correlated Topic Model}
\todo[inline]{ARTM - Additive Regularization for Topic Modeling}
\todo[inline]{\href{http://www.ispras.ru/proceedings/docs/2012/23/isp_23_2012_215.pdf}{Обзор}}
\todo[inline]{\href{http://dwl.kiev.ua/art/index.html}{dwl.kiev.ua - Дмитрия Владимировича Ландэ}}
\todo[inline]{обосновывается необходимость разработки нового или адаптации существующего метода или алгоритма}
\todo[inline]{выводы из обзора (лучше сравнительную таблицу) отсюда актуальность (никто не делал так/улучшаем то-то и то-то)}

%
\section{Формализованное описание проблемы}
\todo[inline]{Необходимая существующая математика}
\todo[inline]{описание входных и выходных данных}
\todo[inline]{Откуда брать данные и какие они бывают}
\todo[inline]{описание критериев сравнения нескольких реализаций метода или алгоритма}

%
\section{// Функциональные требования к }
\todo[inline]{Что мы хотим получить (это и будет "мостиком" к конструкторской)}
