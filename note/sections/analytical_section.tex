\chapter{Аналитический раздел}
\todo[inline]{25 – 30 страниц}

%
\section{Постановка задачи}

\noindent\textbf{Целью} данной работы является разработка метода тематического моделирования для новостей на русском языке.

~\

\noindentДля достижения этой цели необходимо выполнить следующие основные \textbf{задачи}:

\begin{itemize}
    \item \todo{}анализ существующих решений и выбор базового алгоритма тематического моделирования для классификация/категоризация новостей на русском языке;
    \item разработка программного продукта для сбора новостей на русском языке и подготовки данных для последующего анализа;
    \item подбор методов улучшения алгоритма и значений их параметров;
    \item обучение модели;
    \item проведение параметризации метода;
    \item \todo{}проведение апробации метода;
    \item составление рекомендаций о пременимости предложенного метода.
\end{itemize}

%
\section{Задачи тематического моделирования}
\todo[inline]{проводится анализ предметной области}
\todo[inline]{выделяется основной объект исследования}


\noindentЗадачи, для решения которых используется тематическое моделирование разбивают на 2 класса: \textbf{Автоматический анализ текста} и \textbf{систематизация больших объемов информации}.

~\

\noindentВ задачах автоматического анализа текста обычно выделяют следующие направления.

\begin{itemize}
    \item \textbf{Классификация и категоризация документов} - необходимо присвоить каждому документу метку соответствующих классов. Если классы составляют иерархическую структуру, говорят о категоризации.\todo[inline]{Либо сослаться на Воронцова, либо использовать иерархическую классификацию}
    \item \textbf{Автоматическое аннотирование документов} - составление краткого обзора документа на основании использования наиболее важных фраз, используя наиболее важные фразы.
    \item \textbf{Автоматическое реферирование или суммаризация коллекции} - решение предыдущей задачи для большой коллекции документов.
    \item \textbf{Тематическая сегментация документов} - разбиение длинного документа на части с различными темами.
\end{itemize}

~\

\noindentВ задачах систематизации больших объемов информации обычно выделяют следующие направления:

\begin{itemize}
    \item \textbf{Семантический (разведочный) поиск информации} - поиск по коллекции документов на базе тематического моделирования позволяет использовать длинный документ в качестве поискового запроса, а также находить документы, близкие по смыслу, даже если ключевые слова, используемые при поиске, отсутствуют в результатах поиска.
    \item \textbf{Визуализация тематической структуры коллекции} - все задачи, связанные с графическим представлением больших массивов документов.
    \item \textbf{Анализ динамики развития тем} - обычно используется при наличии данных о времени создания документов в коллекции.
    \item \textbf{Тематический мониторинг новых поступлений} - автоматический мониторинг настроенных ресурсов на наличие новых документов, схожих по тематике с настроенным целевым документом.
    \item \textbf{Рекомендация документов пользователям} - создание рекомендательных систем на основании данных о просмотренных пользователем документах и его активности.
\end{itemize}

%
\section{Существующие методы}
\todo[inline]{обзор существующих путей/методов/решений и алгоритмов решения}
\todo[inline]{? Графические модели}
\todo[inline]{? рLDA}
\todo[inline]{?JPM - Join Probabilistic Model, AHMM - Aspect Hidden Markov Model, ATM - Autor-Topic Model, CTM - Correlated Topic Model}
\todo[inline]{\href{http://dwl.kiev.ua/art/index.html}{? dwl.kiev.ua - Дмитрия Владимировича Ландэ}}
\todo[inline]{обосновывается необходимость разработки нового или адаптации существующего метода или алгоритма}
\todo[inline]{выводы из обзора (лучше сравнительную таблицу) отсюда актуальность (никто не делал так/улучшаем то-то и то-то)}
\todo[inline]{рассмотреть математику используемых регуляризаторов} 
\todo[inline]{добавить математику мультимодальности}
%%
\subsection{Основы кластеризации и классификации документов}



Документы представляются векторной моделью (VSM, Vector Space Model). В такой модели каждому слову сопоставляется определенный вес, вычисляемый по весовой функции.

\noindentБазовый вариант весовых функций в таком представлении данных - агрегирующий показатель
$$
TF-IDF(t,d,D)=TF(t,d) \times IDF(t,D),
$$
где

TF — частота слова, которая равна отношению числа вхождения определенного слова к общему числу слов документа.
$$
TF(t,d)={{freq(t,d)}\over{max_{W \in D} freq(w,d)}}
$$

IDF — обратная частота документа, инверсия частоты, с которой определенное слово встречается в документах коллекции.
$$
IDF(t,D)=\log{{|D|}\over{|\{ d \in D:t \in d \}|}}
$$
где

$|D|$ — число документов в коллекции.

$|\{ d \in D:t \in d \}|$ — число документов из коллекции  $D$, в которых встречается  $t$ (когда $n_t \neq 0$).

Выбор основания логарифма в формуле не имеет значения, поскольку изменение основания приводит к изменению веса каждого слова на постоянный множитель, что не влияет на соотношение весов.


В первый раз задача определения и отслеживания тем (TDT, Topic Detection and Tracking) встречается в работе 
"Topic Detection and Tracking Pilot Study. Final Report" [\todo{}]. Темой в этой работе называют событие или действие вместе со всеми непосредственно связанными событиями или действиями. Задачей является извлечение событий.

\todo[inline]{пояснить что такое freq}
\noindentЕще вариант из работы [\todo{}]:
$$
w(t,D)=(1+\log_2{TF(t,D)}) \times {{IDF(t)}\over{||\vec{d}||},}
$$
где $||\vec{d}||$ - номер вектора, представляющего документ $D$.
\noindentЕще варианты модификаций TF-IDF из работ [\todo{}]:
$$
TF'={{TF}\over{TF+0.5+1.5{{l_d}\over{l_{avg}}}}},
$$
где $l_d$ - длинна документа $d$, а $l_{avg}$ - средняя длинна документа.
$$
IDF'={{\log{(IDF)}}\over{\log{(N+1)}}}
$$

Для определения расстояния в таком представлении данных использовались различные метрики: дивергенция Кульбака-Лейблера, косинусная мера и другие. В первых работах для решения таких задач использовались алгоритмы кластеризации - выделение групп близких объектов без обучающей выборке и без сведений о классах: метод К-средних, инкрементальная кластеризация и т. д.[\todo{Клышинский}]Каждый кластер описывал то или иное событие.

Главным недостатком такого подхода является однозначность отношения документ-тема. То есть один документ относится к одной теме (событию). В рассматриваемом ниже примере про новость финансирования спорта будет продемонстрирована, что в одном документе могут затрагиваться сразу две темы и футбол и финансы. При таком подходе эти данные теряются.
\todo[inline]{Добавить блок текста по рекомендации консультанта}

Используется векторное представление текста, как было сказано выше. Координатой документа может быть частота термина или иных конструкций, полученных при анализе текста. Текст подлежит четырем ключевым этапам анализа - морфологическому, синтаксическому, семантическому [\todo{Клышинский}], графематическому. В качество координат документа в данной работе будем рассматривать частоты употребления в нем слов, представленных леммами - начальными формами слова.

\todo[inline]{Дать определение слову Семантика}
%%
\subsection{Латентный семантический анализ (LSA)}
\todo[inline]{дать определение мешка слов}

Dumais et allii. [\todo{}] в 1988 году предложили метод LSA. Суть метода в том, чтобы спроецировать документы и термины в пространство более низкой размерности. Для этого анализируется совместная встречаемость слов (терминов) в документах. Таким образом задача состоит в том, чтобы часто встречающиеся вместе термины были спроецированы в одно и то же измерение семантического пространства.

\todo[inline]{Дописать что надо по минимуму, что бы был понятен PLSA}

%%
\subsection{Вероятностный латентный семантический анализ (PLSA)}

 
В 1999 году Томасом Хофманом был предложен метод вероятностного латентного семантического анализа (PLSA) [\todo{}]. В вероятностных тематических моделях, в отличие от рассмотренных выше методов, сначала задается модель, а после с помощью матрицы слов в документах оцениваются ее скрытые параметры. В связи с этим появляется возможность дообучения моделей и упрощается подбор параметров.

Для лучшего понимания алгоритма рассмотрим детальнее процесс написания новости журналистом. Для начала работы он выбирает тему своей новостной статьи. Это, в свою очередь, влияет на то, какие слова он будет использовать. Очевидно, что если журналист решил написать новость про футбол, то слово <<мяч>> в таком документе появится с большей вероятностью, чем слово <<антиматерия>>. При этом если статья затрагивает финансовую сторону вопроса, то вероятности возникновения слов <<мяч>> и слово <<бюджет>> могут сравняться. В таком случае мы можем сказать,  что такая новость имеет минимум две темы - <<спорт>> и <<финансы>>, которые в свою очередь и породили слова <<мяч>> и <<бюджет>>. 

Продолжая эту аналогию, можно представить любую новость как смесь различных тем, которые в свою очередь породили слова. 

<<процесс порождения текстового документа вероятностной тематической моделью.png>>
\todo[inline]{Вставить картинку}

\noindentПриняты следующие допущения.

\begin{itemize}
    \item Порядок слов в документе не важен (bag of words).
    \item Слова в документах генерируются темой, а не самим документом.
    \item Порядок документов в коллекции не важен.
    \item Каждое отношение документ-слово $(d,w)$ связано с некоторой темой $t \in T$.
    \item Коллекция представляет собой последовательность троек документ-слово-тема $(d,w,t)$.
    \item В теме невелико число образующих слов.
    \item В документе используется небольшое число тем.
\end{itemize}

~\

\noindentПусть


 $D$ - коллекция документов размера $n_d$ с документами $d$,

 $W$ - словарь терминов размера $n_w$ со словами $w$,

 $T$ - список тем размера размера $n_t$ с темами $t$,

 $n_{dw}$ - количество использований слова $w$ в документе $d$,

 каждый документ состоит из слов: $d \subset W$,

 $p(w|d)$ - вероятность появления слова $w$ в документе $d$,

 $p(w|t)$ - вероятность появления слова $w$ в теме $t$,

 $p(t|d)$ - вероятность появления темы $t$ в документе $d$,

 $\hat{p}(w|d) = {{n_dw}\over{n_d}}$ - наблюдаемая частота слова $w$ в документе $d$.


~\

\noindentТребуется найти параметры вероятностной порождающей тематической модели, то есть представить вероятность появления слов в документе $p(w|d)$ в виде:
$$
p(w|d) = \sum_{t \in T}{ p(w|t) p(t|d) }.
$$
\noindentЗапишем вероятности $p(w|t)$ в матрицу $\Phi=(\phi_{wt})$, а вероятности $p(t|d)$ - в матрицу $\Theta=(\theta_{td})$. Тогда вероятность появления слов в документе можно представить в виде матричного разложения:
$$
p(w|d) = \sum_{t \in T}{ \phi_{wt} \theta_{td} }.
$$
<<матричное разложение.png>>
\todo[inline]{Вставить картинку}

\noindentТо есть решается задача, обратная к генерации текста (работе журналиста). Необходимо по имеющийся коллекции документов понять, какими распределениями матриц $\phi_{wt}$ и $\theta_{td}$ она могла быть получена.

\todo[inline]{Понятие стохастической матрицы}

\noindentТеперь, воспользовавшись принципом максимума правдоподобия с ограничениями на элементы стохастических матриц, если максимизировать  логарифм правдоподобия,получается:

$$ 
\begin{cases}
    \sum_{d \in D} \sum_{w \in d} n_{dw} \ln{\sum_{t \in T} \phi_{wt} \theta_{td} } \rightarrow max_{\Phi,\Theta};\\
    \sum_{w \in W}\phi_{wt} = 1; &\phi_{wt} \ge 0;\\
    \sum_{t \in T}\theta_{td} = 1; &\theta_{td} \ge 0.\\
\end{cases}
$$

%%
\subsection{Латентное размещение Дирихле (LDA)}

\noindentЗадача в таком виде поставлена не корректно так как существует больше одного решения этой системы:
$$
\Phi\Theta = (\Phi S)(S^{-1}\Theta)=\Phi'\Theta'.
$$
\noindentТо есть результаты будут зависеть от стартовых значений параметров модели и при кадом обучении будут различаться. Но так же это означает, что есть возможность модифицировать алгоритм, сужая пространство решений. Введем для этого критерий регуляризации $R(\Phi,\Theta)$ - некоторый функционал, соответствующий прикладной задаче, для которой обучается модель. Рассмотрим задачу максимизации регуляризованного правдоподобия:

$$ 
\begin{cases}
    \sum_{d \in D} \sum_{w \in d} n_{dw} \ln{\sum_{t \in T} \phi_{wt} \theta_{td} } + R(\Phi,\Theta) \rightarrow max_{\Phi,\Theta};\\
    \sum_{w \in W}\phi_{wt} = 1; &\phi_{wt} \ge 0;\\
    \sum_{t \in T}\theta_{td} = 1; &\theta_{td} \ge 0.\\
\end{cases}
$$

В 2003 году Дэвидом Блеем, Эндрю Энджи и Маклом Джорданом был предложен метод латентного размещения Дирихле (LDA) [\todo{}]. На данный момент это одна из самых цитируемых статей по тематическому моделированию. Они предложили решать задачу со следующим регуляризатором:
$$
R(\Phi,\Theta) = \sum_{t,w}{(\beta_w-1)\ln{\phi_{wt}}} + \sum_{d,t}{(\alpha_t-1)\ln{\theta_{td}}},
$$
$$
\beta_w > 0, 
$$
$$
\alpha_t > 0,
$$
где $\beta_w$ и $\alpha_t$ - параметры регуляризатора.

Для понимания метода введем понятие дивергенции Кульбака-Лейблера для дискретных распределений.\noindentПусть даны два дискретных распределения $P=(p_i)_{i=1}^n$ и $Q=(q_i)_{i=1}^n$, тогда дивергенция Кульбака-Лейблера выражается так
$$
KL(P||Q)=\sum_i{p_i \log{{p_i}\over{q_i}}}.
$$
Дивергенция Кульбака-Лейблера обладает следующими свойствами.

\begin{itemize}
    \item неотрицательность:
        $$
        KL(P||Q)\ge 0;
        $$
        $$
        KL(P||Q)=0 \Leftrightarrow P=Q
        $$
    \item несимитричность:
        $$
        KL(P||Q)\neq KL(Q||P)
        $$
\end{itemize}

~\

Дивергенция Кульбака-Лейблера связана с максимумом правдоподобия:
$$
\sum_{i=1}^{n}p_i\ln{p_i \over {q_i(\alpha)}} \rightarrow \underset{\alpha}{min} \Leftrightarrow \sum_{i=1}^{n}p_i\ln{q_i(\alpha)} \rightarrow \underset{\alpha}{max}
$$

Минимизация дивергенции Кульбака-Лейблера эквивалентна максимизации правдоподобия. Пусть $P$ - эмпирическое распределение. $Q$ - параметрическая модель распределения с параметром $\alpha$. При минимизации дивергенции Кульбака-Лейблера (максимизации правдоподобия) определяется такое значение $\alpha$, при котором $P$ как можно лучше соответствует модели.

Пусть $\beta=(\beta_w)$ - некоторый вектор над словарем $W$ со словами $w$.

При $\beta_w>1$ вероятность $\phi_{wt}$ этого слова по темам будет сглаживаться, приближаясь к $\beta_w^+$ : 
$$
KL(\beta^+||\phi_t) \rightarrow min,
$$
$$
\beta_w^+=\underset{w \in W}{norm}(\beta_w-1)
$$
При $\beta_w<1$ значение $\phi_{wt}$ наоборот будут разреживаться, удаляясь от $\beta_w^-$ к нулю : 
$$
KL(\beta^-||\phi_t) \rightarrow max,
$$
$$
\beta_w^-=\underset{w \in W}{norm}(1-\beta_w)
$$
то есть в матрице $\Phi$ будет больше нулевых элементов или близких к нулю.

%%
\subsection{Аддитивная регуляризация тематических моделей (ARTM)}

Неединственность решения максимизации регуляризованного правдоподобия позволяет накладывать сразу несколько ограничений на модель, этот метод называется аддитивной регуляризацией тематических моделей (ARTM).

То есть
$$
\sum_{d,w}{n_{dw} \ln{\sum_t\phi_{wt} \theta_{td}}} + \sum_{i=1}^k \tau_i R_i(\Phi,\Theta) \rightarrow \underset{\Phi\Theta}{max}
$$
где $\tau_i$ - коэффициенты регуляризации, а $R_i(\Phi,\Theta)$ - регуляризаторы.

При таком подходе возникает проблема поиска коэффициентов, которая обычно решается добавлением регуляризаторов в модель по одному и оптимизации соответствующих коэффициентов в ходе пробных запусков моделей.

%%
\subsection{Решение задачи максимизации регуляризованного правдоподобия}
\todo[inline]{добавить ссылку на обснование EM алгоритма}
Решение задачи в общем виде аналитическими методами слишком сложно. Однако, если выбирать гладкие регуляризаторы, то можно воспользоваться условием Крауша-Куна-Таккера. Получится система уравнений:
$$
\begin{cases}
    p_{tdw} = \underset{t \in T}{norm}(\phi_{wt} \theta_{td}) \\
    \phi_{wt} = \underset{w \in W}{norm}\bigg( \underset{d \in D}\sum{n_{dw} p_{tdw} + \phi_{wt} {{\partial R} \over {\partial \phi_{wt}}}} \bigg) \\
    \theta_{td} = \underset{t \in T}{norm}\bigg( \underset{w \in d}\sum{n_{dw} p_{tdw} + \theta_{td} {{\partial R} \over {\partial \theta_{td}}}} \bigg) \\
\end{cases}
$$
где
$$
\underset{t \in T}{norm}(x_t) = {{max\{ x_t, 0 \}} \over {\underset{s \in T}\sum{ max\{ x_s, 0 \}} }}
$$

Такую систему можно решить численным методом простых итераций. В данном случае его называют  ЕМ-алгоритм.

Для получения результата необходимо итерационно выполнять Е-шаг и М-шаг до достижения требуемой точности.

Е-шаг :
$$
p_{tdw}=\underset{t \in T}{norm}(\phi_{wt} \theta_{td})
$$

М-шаг : 
$$
\phi_{wt} = \underset{w \in W}{norm}\bigg( \underset{d \in D}\sum{n_{dw} p_{tdw} + \phi_{wt} {{\partial R} \over {\partial \phi_{wt}}}} \bigg)
$$
$$
\theta_{td} = \underset{t \in T}{norm}\bigg( \underset{w \in d}\sum{n_{dw} p_{tdw} + \theta_{td} {{\partial R} \over {\partial \theta_{td}}}} \bigg)
$$

Этот процесс можно организовать параллельно, если обновлять матрицу   $\Phi$ по порциям, после анализа очередного пакета документов. Обычно уже после просмотра нескольких первых десятков тысяч документов матрица $\Phi$ получается уже устоявшиеся и остается только тематизировать остальные документы 


%%
\subsection{Выбор алгоритма}
\todo[inline]{Добавить выбор алгоритма}

В данной работе рассматривается задача \todo{}классификации и категоризации документов. В качестве документов выступают новости на русском языке. Необходимо с помощью выбранного метода и способов его усовершенствования разбить коллекцию новостей на темы, интерпретируемые человеком и получить возможность оценивать новый документ (новость) на принадлежность этим темам.

Особенностью тематического моделирования является возможность не использовать в процессе построения модели размеченные данные. То есть темы, на которые разбивается коллекция, также создаются в процессе формирования модели. 

%%
\section{Формализованное описание проблемы}
\todo[inline]{Откуда брать данные и какие они бывают}
\todo[inline]{описание критериев сравнения нескольких реализаций метода или алгоритма}

\noindentВходные данные:

\begin{itemize}
    \item коллекция новостей на русском языке на разные темы в сети интернет.
\end{itemize}

~\

\noindentВыходные данные:

\begin{itemize}
    \item обученная тематическая модель с настроенными регуляризаторами;
    \item список тем с образующими их словами;
    \item \todo{}названия тем.
\end{itemize}

~\

\noindentПолучение данных:

\begin{itemize}
    \item парсинг новостных агрегаторов;
    \item парсинг крупных новостных сайтов.
\end{itemize}

~\

\noindentПодготовка данных:

\begin{itemize}
    \item удаление форматирования текста;
    \item исправление опечаток;
    \item слияние слишком коротких текстов;
    \item выделение терминов;
    \item приведение слов к нормальной форме (лемматизация);
    \item удаление слишком частых слов;
    \item удаление слишком редких слов.
\end{itemize}

~\



%
\section{Функциональные требования}
\todo[inline]{Что мы хотим получить (это и будет "мостиком" к конструкторской)}

Для решения задачи классификации и категоризации новостей на русском языке необходимо следующее: 

\begin{itemize}
    \item собирать новости из ресурсов сети Интернет;
    \item преобразовывать их в необходимый формат;
    \item создавать и обучать модель;
    \item провести параметризацию: подобрать наилучший комплект регуляризаторов, их параметров и коэффициентов;
    \item иметь возможность последующего повторного использования и дообучения модели.
\end{itemize}

\todo{}

