\chapter{Аналитический раздел}
\todo[inline]{25 – 30 страниц}

%
\section{// цель, перечисляются задачи, которые необходимо решить для достижения этой цели}

%
\section{// проводится анализ предметной области и выделяется основной объект исследования}

%
\section{// обзор существующих путей/методов/решений и алгоритмов решения}
\todo[inline]{Классификация}
\todo[inline]{Кластеризация}
\todo[inline]{PLSA}
\todo[inline]{ARTM}
\todo[inline]{\href{http://dwl.kiev.ua/art/index.html}{dwl.kiev.ua - Дмитрия Владимировича Ландэ}}


%
\section{// обосновывается необходимость разработки нового или адаптации существующего метода или алгоритма}
\todo[inline]{выводы из обзора (лучше сравнительную таблицу) отсюда актуальность (никто не делал так/улучшаем то-то и то-то)}

%
\section{// формализованное описание проблемы предметной области}
\todo[inline]{Необходимая существующая математика}

%%
\subsection{описание входных и выходных данных}
\todo[inline]{Откуда брать данные и какие они бывают}

%%
\subsection{описание критериев сравнения нескольких реализаций метода или алгоритма}

%%
\subsection{описание функциональных требований к разрабатываемому программному
обеспечению}
\todo[inline]{Что мы хотим получить (это и будет "мостиком" к конструкторской)}
