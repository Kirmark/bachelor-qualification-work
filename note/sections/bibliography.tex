\chapter{Список источников}

%
\section{// Разобрать}
\todo[inline]{пройтись по всем упоминаниям в тексте и добавить сюда}
\todo[inline]{проверить, что сюда попали статьи, которые я распечатал для себя}
\todo[inline]{\href{https://datafest.us18.list-manage.com/track/click?u=acc56a45f4f4d03aa67f9cd69&id=10e312a2ae&e=dd724b189a}{Ссылка на записи с datafest}}
\todo[inline]{Воронцов - книги и лекции}
\todo[inline]{Ученики Воронцова - доклады и статьи}
\todo[inline]{Анастасия Янина - работала с Воронцовым - посмотреть ее доклады и статьи}
\todo[inline]{Потапенко Анна - работала с Воронцовым - посмотреть ее доклады и статьи}
\todo[inline]{"Диалог" - NLP Конференция}
\todo[inline]{\href{https://www.coursera.org/learn/unsupervised-learning/supplement/suSWG/noutbuk-iz-diemonstratsii-ispol-zovaniia-bigartm}{курсы на курсере}}
\todo[inline]{\href{http://dwl.kiev.ua/art/index.html}{dwl.kiev.ua - Дмитрия Владимировича Ландэ}}
\todo[inline]{\href{http://www.ispras.ru/proceedings/docs/2012/23/isp_23_2012_215.pdf}{Обзор}}
\todo[inline]{Topic Detection and Tracking Pilot Study. Final Report.}

%
\section{// Датасеты}
\todo[inline]{перепроверить, что тут есть все датасеты из технологическй части}
\todo[inline]{25 500 новостей (там суммарно 9 000 000 слов - я посчитал) за все время существования media.zone (я сам написал парсер, могу его же натравить на любой другой новостной ресурс) - уже скачены и лежат на моем компьютере}
\todo[inline]{\href{http://www.statmt.org/wmt15/translation-task.html}{statmt.org - это не совсем подходит нам, тут новости короткие совсем. Но тоже скачал на всякий случай поиграться - тут суммарно 8,4 гигабайта чистого текста - уже скачены и лежат на моем компьютере}}
\todo[inline]{\href{https://webhose.io/free-datasets/russian-news-articles/}{webhose.io - 290 000 новостей - уже скачены и лежат на моем компьютере}}
\todo[inline]{Можно сделать сервис на РИА новости}
\todo[inline]{Можно сделать сервис на агрегаторы новостей}
\todo[inline]{убрать нумерацию перед разделом}