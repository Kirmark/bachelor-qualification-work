\chapter{Технологический раздел}
\todo[inline]{20 - 25 страниц}
\todo[inline]{обоснованный выбор средств программной реализации}
\todo[inline]{описание основных (нетривиальных) моментов разработки}

%
\section{Выбор средств программной реализации и разработка}

%%
\subsection{Создание базы данных}

Для данной работы рассматривается несколько самых известных реализаций реляционных баз данных:

\begin{itemize}
    \item MySQL
    \item SQLite
    \item PostrgreSQL
\end{itemize}

%%%
\subsubsection{MySQL}

Решение от компании Oracle. Очень популярное и мощное решение для малых и средних приложений, распространяемое под лицензией \href{https://ru.wikipedia.org/wiki/GNU_General_Public_License}{GNU General Public License}. Преимущества этого решения - популярность и богатый функционал. Из недостатков можно отметить требовательность к ПО и относительно медленная разработка.

%%%
\subsubsection{SQLite}

Компактная встраиваемая СУБД. Движок SQLite представляет собой библиотеку, а не отдельно работающий процесс. При работе с этой СУБД обращения происходят напрямую к файлам. Среди недостатков можно отметить небольшое количество типов данных, доступных по умолчанию, отсутствие системы пользователей. Среди преимуществ хранение всей базы одним файлом.

%%%
\subsubsection{PostrgreSQL}

Самое профессиональное из всех трех рассмотренных решений. Обладает богатым функционалом. PostrgreSQL это не только реляционная СУБД, но также и объектно-ориентированная. К недостаткам можно отнести низкую производительность на простых операциях.

%%%
\subsubsection{Выбр СУБД}

Исходя из технических требований для этой работы выбор был остановлен на SQLite. Использование данного решения позволяет хранить все в одном файле и упрощает стартовую настройку решения. Ограниченность функционала и типов данных не будет проблемой в связи с простой структурой данных.

В качестве дополнительного функционала был реализован подсчет рейтинга страниц, который становится тем больше чем больше ссылок ведет на рассматриваемую страницу. Данный подход часто используется при сортировке страниц в поисковой выдаче. Эти данные могут пригодиться для процесса сохранения html-файлов. Можно модифицировать решение и в первую очередь скачивать страницы с наибольшим рейтингом.

%%
\subsection{Сбор данных}



%%
\subsection{Обработка данных}

%%%
\subsubsection{Обработка дкумента}

%%%
\subsubsection{Подготовка коллекции}

%%
\subsection{Обучение модели}

%%
\subsection{Использование модели}

%%%
\subsubsection{Оценка документа}

%%%
\subsubsection{Дообучение}

%%
\subsection{Оценка модели}

%
\section{Тестирование}
\todo[inline]{методики тестирования созданного программного обеспечения}

%
\section{Подготовка к запуску}
\todo[inline]{информация, необходимая для сборки и запуска разработанного программного обеспечения}