\chapter{Исследовательский раздел}

%
\section{Апробация метода}

%%
\subsection{Подготовка данных}

Для исследования было принято решение использовать 3 блока данных:

\begin{itemize}
    \item 23 000 записей с сайта zona.media
    \item 1 000 000 записей с сайта ria.ru
    \item 24 000 случайно выбранных записей из 1 000 000 с сайта ria.ru
\end{itemize}

~\

Подготовка двух блоков данных по 23 и 24 тысячи записей происходила в один поток. Для обработки 1 миллиона записей была использована многопоточность. Записи были обработаны в 16 парааллельных потоков.

%%
\subsection{Подбор базовых значений коэффициентов регуляризации}

Что бы получить первые результаты исследования необходимо определить хотя бы примерно центры диапазонов коэффициентов при регуляризаторах для последующего их уточнения. После каждой попытки анализируется результат. Проверяется достиг ли исследователь своей цели по соответствующему параметру и не выродилась ли при этом модель.

Так как регуляризаторы добавляются последовательно, поиск базовых значений так же реализован последовательно. Сначала на коллекции обучается модель PLSA до тех пор, пока модель не сойдется. После этого добавляется первый регуляризатор, разреживающий матрицу слова-темы $\Phi$.

После нескольких пробных

%%
\subsection{Определение оптимального количества тем}

%%
\subsection{Корректировка коэффициентов регуляризации}

%
\section{Анализ результатов}

%
\section{Рекомендации}
