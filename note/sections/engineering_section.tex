\chapter{Конструкторский раздел}


\section{// Алгоритм сбора данных}

\todo[inline]{Мой написанный код для парсинга}
\todo[inline]{Уже предварительно собранные открытые данные}
\todo[inline]{https://newspaper.readthedocs.io/en/latest/ - возможный инструмент для парсинга}


\section{// Алгоритм анализа}

\todo[inline]{Базовый алгоритм: ARTM (bigartm.readthedocs.io)}
\todo[inline]{Предобработка текста: лемматизация, удаление стоп-слов, ngrams}
\todo[inline]{Используем модальности (дата публикации, ссылки на другие документы, авторы)}
\todo[inline]{Используем производные от статьи данные по различным алгоритмам (записываем в модальности) - алгоритмы еще не выбраны}


\section{// Что делаем}

\todo[inline]{Можно попробовать обучаться на месяце/неделе/дне (и это в теории можно вынести в экперимент) и выдавать как меняются темы}
\todo[inline]{решить иерархически ли хотим строить темы или многое ко многим}


\section{// Тесты}

\todo[inline]{Разбиение на 2 части и замеры разницы оценки - устойчивость - Через предложение разбивать статью можно попробовать}
\todo[inline]{Толока - описание теста - выбрать лишнее слово, подумать что еще можно}

