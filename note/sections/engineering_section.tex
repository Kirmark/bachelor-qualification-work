\chapter{Конструкторский раздел}
\todo[inline]{25 – 30 страниц}

%
\section{// обосновать последовательность этапов выполнения}

%
\section{// Алгоритм сбора данных}
\todo[inline]{как будем извлекать данные (без кода пока)}
\todo[inline]{Мой написанный код для парсинга}
\todo[inline]{Уже предварительно собранные открытые данные}
\todo[inline]{https://newspaper.readthedocs.io/en/latest/ - возможный инструмент для парсинга}
\todo[inline]{25 500 новостей (там суммарно 9 000 000 слов - я посчитал) за все время существования media.zone (я сам написал парсер, могу его же натравить на любой другой новостной ресурс) - уже скачены и лежат на моем компьютере}
\todo[inline]{\href{http://www.statmt.org/wmt15/translation-task.html}{statmt.org - это не совсем подходит нам, тут новости короткие совсем. Но тоже скачал на всякий случай поиграться - тут суммарно 8,4 гигабайта чистого текста - уже скачены и лежат на моем компьютере}}
\todo[inline]{\href{https://webhose.io/free-datasets/russian-news-articles/}{webhose.io - 290 000 новостей - уже скачены и лежат на моем компьютере}}
\todo[inline]{Можно сделать сервис на РИА новости}
\todo[inline]{Можно сделать сервис на агрегаторы новостей}

%
\section{// Алгоритм анализа}
\todo[inline]{разработка метода}
\todo[inline]{Базовый алгоритм: ARTM (bigartm.readthedocs.io)}
\todo[inline]{Предобработка текста: лемматизация, удаление стоп-слов, ngrams}
\todo[inline]{Используем модальности (дата публикации, ссылки на другие документы, авторы)}
\todo[inline]{Используем производные от статьи данные по различным алгоритмам (записываем в модальности) - алгоритмы еще не выбраны}
\todo[inline]{IDEF0 метода}

%
\section{// ? Что делаем}
\todo[inline]{Можно попробовать обучаться на месяце/неделе/дне (и это в теории можно вынести в экперимент) и выдавать как меняются темы}
\todo[inline]{решить иерархически ли хотим строить темы или многое ко многим}

%
\section{// Оценка}
\todo[inline]{как будем оценивать (без кода)}
\todo[inline]{Разбиение на 2 части и замеры разницы оценки - устойчивость - Через предложение разбивать статью можно попробовать}
\todo[inline]{Толока - описание теста - выбрать лишнее слово, подумать что еще можно}

%
\section{// Требования к программе}
