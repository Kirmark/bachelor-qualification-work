\chapter*{Реферат}

Отчет страниц, 4 части, 8 рисунков, 2 таблицы, 10 источников.

Объектом исследования является метод тематического моделирования в применении к новостям на русском языке.

Цель работы: разработка метода тематического моделирования для новостей на русском языке.

В данной работе разрабатывается метод тематического анализа новостей на русском языке. Реализовано ПО для  начиная первого этапа - сбора данных из сети Интернет. Рассмотрен процесс обработки и подготовки данных, создания модели. Сравниваются результаты различных модификаций, и создаются рекомендации для применения.


Расчетно-пояснительная записка содержит аналитический, конструкторский, технологический и исследовательский разделы.

В аналитическом разделе детально изучена предметная область. Проведен анализ принципов работы существующих методов тематического моделирования.

В конструкторском разделе создана структура данных для хранения коллекций новостей. Описаны принятые проектные решения. Рассмотрены варианты и выбраны способы реализации решения. Выделены функциональные требования к решению.

В технологическом разделе описан выбор средств разработки, описаны нетривиальные моменты реализации. 

В исследовательском разделе продемонстрирован и проанализирован процесс классификации новостей на нескольких коллекциях. Разобран процесс подбора коэффициентов при регуляризации. Проведен сравнительный анализ обученных моделей.

Поставленная цель работы достигнута: разработан метод тематического моделирования для новостей на русском языке.