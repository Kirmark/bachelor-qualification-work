Реферат
\todo[inline]{Сделать что бы тут был заголовок, но не включался в оглавление}

Отчет страниц, 4 части, \todo{} рисунков\todo{} , \todo{} таблиц\todo{} , \todo{} источник\todo{} .

Объектом исследования является метод тематического моделирования в применении к новостям на русском языке.

В данной работе разрабатывается метод тематического анализа новостей на русском языке, начиная с самого первого этапа - сбора данных из сети интернет. Рассмотрен процесс обработки и подготовки данных, создания модели. Сравниваются результаты различных модификаций и создаются рекомендации для применения.

Цель работы разработка метода тематического моделирования для новостей на русском языке.

Расчетно-пояснительная записка содержит аналитический, конструкторский, технологический и исследовательский разделы.

В аналитическом разделе детально изучена предметная область. Проведен анализ работы существующих способов тематического моделирования. Создано формальное описание проблемы.

В конструкторском разделе создана структура данных для хранения коллекции новостей. Рассмотрены варианты и выбраны способы реализации решения. Описаны функциональные требования к решению.

В технологическом разделе описан выбор средств разработки, описаны нетривиальные моменты реализации. Созданы технические требования к решению.

В исследовательском разделе продемонстрирован процесс классификации новостей на нескольких коллекциях. Разобран процесс подбора коэффициентов при регуляризации. Проведен сравнительный анализ.

Поставленная цель работы достигнута: разработан метод тематического моделирования для новостей на русском языке.