%%%        Подключение пакетов                 %%%
\usepackage{ifthen}                 % добавляет ifthenelse
%%% Инициализирование переменных, не трогать!  %%%
\newcounter{intvl}
\newcounter{otstup}
\newcounter{contnumeq}
\newcounter{contnumfig}
\newcounter{contnumtab}
\newcounter{pgnum}
\newcounter{bibliosel}
\newcounter{chapstyle}
\newcounter{headingdelim}
\newcounter{headingalign}
\newcounter{headingsize}
\newcounter{tabcap}
\newcounter{tablaba}
\newcounter{tabtita}
%%%%%%%%%%%%%%%%%%%%%%%%%%%%%%%%%%%%%%%%%%%%%%%%%%


%%% Изображения %%%
\graphicspath{{images/}{Dissertation/images/}}         % Пути к изображениям

\LoadInterface {titlesec}                   % Подгружаем интерфейсы для дополнительных опций управления некоторыми пакетами

%%% Блок управления параметрами для выравнивания заголовков в тексте %%%
\newlength{\otstuplen}
\setlength{\otstuplen}{\theotstup\parindent}
\ifthenelse{\equal{\theheadingalign}{0}}{% выравнивание заголовков в тексте
    \newcommand{\hdngalign}{\filcenter}                % по центру
    \newcommand{\hdngaligni}{\hfill\hspace{\otstuplen}}% по центру
}{%
    \newcommand{\hdngalign}{\filright}                 % по левому краю
    \newcommand{\hdngaligni}{\hspace{\otstuplen}}      % по левому краю
} % В обоих случаях вроде бы без переноса, как и надо (ГОСТ Р 7.0.11-2011, 5.3.5)

%%% Оглавление %%%
\renewcommand{\cftchapdotsep}{\cftdotsep}                % отбивка точками до номера страницы начала главы/раздела
\renewcommand{\cfttoctitlefont}{\hdngaligni\fontsize{14pt}{16pt}\selectfont\bfseries}% вместе со следующей строкой
\renewcommand{\cftaftertoctitle}{\hfill}                 % устанавливает заголовок по центру
\setlength{\cftbeforetoctitleskip}{-1.4\curtextsize}     % Поскольку этот заголовок всегда является первым на странице, то перед ним отделять пустым тройным интервалом не следует. Независимо от основного шрифта, в этом случае зануление (почти) происходит при -1.4\curtextsize.
\setlength{\cftaftertoctitleskip}{\theintvl\curtextsize} % Если считаем Оглавление заголовком, то выставляем после него тройной интервал через наше определённое значение

%% Переносить слова в заголовке не допускается (ГОСТ Р 7.0.11-2011, 5.3.5). Заголовки в оглавлении должны точно повторять заголовки в тексте (ГОСТ Р 7.0.11-2011, 5.2.3). Прямого указания на запрет переносов в оглавлении нет, но по той же логике невнесения искажений в смысл, лучше в оглавлении не переносить:
\cftsetrmarg{2.55em plus1fil}                       %To have the (sectional) titles in the ToC, etc., typeset ragged right with no hyphenation
\renewcommand{\cftchappagefont}{\normalfont}        % нежирные номера страниц у глав в оглавлении
\renewcommand{\cftchapleader}{\cftdotfill{\cftchapdotsep}}% нежирные точки до номеров страниц у глав в оглавлении
%\renewcommand{\cftchapfont}{}                       % нежирные названия глав в оглавлении

\ifthenelse{\theheadingdelim > 0}{%
    \renewcommand\cftchapaftersnum{.\ }   % добавляет точку с пробелом после номера раздела в оглавлении
}{%
\renewcommand\cftchapaftersnum{\quad}     % добавляет \quad после номера раздела в оглавлении
}
\ifthenelse{\theheadingdelim > 1}{%
    \renewcommand\cftsecaftersnum{.\ }    % добавляет точку с пробелом после номера подраздела в оглавлении
    \renewcommand\cftsubsecaftersnum{.\ } % добавляет точку с пробелом после номера подподраздела в оглавлении
}{%
\renewcommand\cftsecaftersnum{\quad}      % добавляет \quad после номера подраздела в оглавлении
\renewcommand\cftsubsecaftersnum{\quad}   % добавляет \quad после номера подподраздела в оглавлении
}

\ifthenelse{\equal{\thepgnum}{1}}{%
    \addtocontents{toc}{~\hfill{Стр.}\par}% добавить Стр. над номерами страниц
}

%%% Оформление названий глав %%%
%% настройки заголовка списка рисунков
\renewcommand{\cftloftitlefont}{\hdngaligni\fontsize{14pt}{16pt}\selectfont\bfseries}% вместе со следующей строкой
\renewcommand{\cftafterloftitle}{\hfill}                                             % устанавливает заголовок по центру
\setlength{\cftbeforeloftitleskip}{-1.5\curtextsize}     % Поскольку этот заголовок всегда является первым на странице, то перед ним отделять пустым тройным интервалом не следует. Независимо от основного шрифта, в этом случае зануление (почти) происходит при -1.5\curtextsize.
\setlength{\cftafterloftitleskip}{\theintvl\curtextsize} % выставляем после него тройной интервал через наше определённое значение

%% настройки заголовка списка таблиц
\renewcommand{\cftlottitlefont}{\hdngaligni\fontsize{14pt}{16pt}\selectfont\bfseries}% вместе со следующей строкой
\renewcommand{\cftafterlottitle}{\hfill}                                             % устанавливает заголовок по центру
\setlength{\cftbeforelottitleskip}{-1.5\curtextsize}     % Поскольку этот заголовок всегда является первым на странице, то перед ним отделять пустым тройным интервалом не следует. Независимо от основного шрифта, в этом случае зануление (почти) происходит при -1.5\curtextsize.
\setlength{\cftafterlottitleskip}{\theintvl\curtextsize} % выставляем после него тройной интервал через наше определённое значение

\ifnum\curtextsize>\bigtextsize     % Проверяем условие использования базового шрифта 14 pt
\setlength{\headheight}{17pt}       % Исправляем высоту заголовка
\else
\setlength{\headheight}{15pt}       % Исправляем высоту заголовка
\fi

%%% Колонтитулы %%%
% Порядковый номер страницы печатают на середине верхнего поля страницы (ГОСТ Р 7.0.11-2011, 5.3.8)
\makeatletter
\let\ps@plain\ps@fancy              % Подчиняем первые страницы каждой главы общим правилам
\makeatother
\pagestyle{fancy}                   % Меняем стиль оформления страниц
\fancyhf{}                          % Очищаем текущие значения
\fancyfoot[C]{\thepage}             % Печатаем номер страницы на середине верхнего поля
\renewcommand{\headrulewidth}{0pt}  % Убираем разделительную линию

%%% Оформление заголовков глав, разделов, подразделов %%%
%% Работа должна быть выполнена ... размером шрифта 12-14 пунктов (ГОСТ Р 7.0.11-2011, 5.3.8). То есть не должно быть надписей шрифтом более 14. Так и поставим.
%% Эти установки будут давать одинаковый результат независимо от выбора базовым шрифтом 12 пт или 14 пт
\titleformat{\chapter}[block]                                % default display;  hang = with a hanging label. (Like the standard \section.); block = typesets the whole title in a block (a paragraph) without additional formatting. Useful in centered titles
        {\hdngalign\fontsize{14pt}{16pt}\selectfont\bfseries}% 
        %\fontsize{<size>}{<skip>} % второе число ставим 1.2*первое, чтобы адекватно отрабатывали команды по расчету полуторного интервала (домножая разные комбинации коэффициентов на этот)
        {\thechapter\cftchapaftersnum}                       % Заголовки в оглавлении должны точно повторять заголовки в тексте (ГОСТ Р 7.0.11-2011, 5.2.3).
        {0em}% отступ от номера до текста
        {}%

\titleformat{\section}[block]                                % default hang;  hang = with a hanging label. (Like the standard \section.); block = typesets the whole title in a block (a paragraph) without additional formatting. Useful in centered titles
        {\hdngalign\fontsize{14pt}{16pt}\selectfont\bfseries}% 
        %\fontsize{<size>}{<skip>} % второе число ставим 1.2*первое, чтобы адекватно отрабатывали команды по расчету полуторного интервала (домножая разные комбинации коэффициентов на этот)
        {\thesection\cftsecaftersnum}                        % Заголовки в оглавлении должны точно повторять заголовки в тексте (ГОСТ Р 7.0.11-2011, 5.2.3).
        {0em}% отступ от номера до текста
        {}%

\titleformat{\subsection}[block]                             % default hang;  hang = with a hanging label. (Like the standard \section.); block = typesets the whole title in a block (a paragraph) without additional formatting. Useful in centered titles
        {\hdngalign\fontsize{14pt}{16pt}\selectfont\bfseries}% 
        %\fontsize{<size>}{<skip>} % второе число ставим 1.2*первое, чтобы адекватно отрабатывали команды по расчету полуторного интервала (домножая разные комбинации коэффициентов на этот)
        {\thesubsection\cftsubsecaftersnum}                  % Заголовки в оглавлении должны точно повторять заголовки в тексте (ГОСТ Р 7.0.11-2011, 5.2.3).
        {0em}% отступ от номера до текста
        {}%

\ifthenelse{\equal{\thechapstyle}{1}}{%
    \sectionformat{\chapter}{% Параметры заголовков разделов в тексте
        label=\chaptername\ \thechapter\cftchapaftersnum,
        labelsep=0em,
    }
    %% Следующие две строки: будет вписано слово Глава перед каждым номером раздела в оглавлении   
    \renewcommand{\cftchappresnum}{\chaptername\ }
    \setlength{\cftchapnumwidth}{\widthof{\cftchapfont\cftchappresnum\thechapter\cftchapaftersnum}}
}%

%% Интервалы между заголовками
% На эти величины titlespacing множит через *
\beforetitleunit=\curtextsize% привязались к нашему размеру шрифта
\aftertitleunit=\curtextsize% привязались к нашему размеру шрифта

% Счётчик intvl и длина \otstup определены в файле setup
\titlespacing{\chapter}{\theotstup\parindent}{-1.7em}{*\theintvl}       % Заголовки отделяют от текста сверху и снизу тремя интервалами (ГОСТ Р 7.0.11-2011, 5.3.5). Поскольку название главы всегда является первым на странице, то перед ним отделять пустым тройным интервалом не следует. Независимо от основного шрифта, в этом случае зануление происходит при -1.7em.
\titlespacing{\section}{\theotstup\parindent}{*\theintvl}{*\theintvl}
\titlespacing{\subsection}{\theotstup\parindent}{*\theintvl}{*\theintvl}
\titlespacing{\subsubsection}{\theotstup\parindent}{*\theintvl}{*\theintvl}

%%% Блок дополнительного управления размерами заголовков
\ifthenelse{\equal{\theheadingsize}{1}}{% Пропорциональные заголовки и базовый шрифт 14 пт
    \renewcommand{\cfttoctitlefont}{\hdngaligni\Large\bfseries} % Исправляем размер заголовка оглавления
    \setlength{\cftbeforetoctitleskip}{-1.2\curtextsize}        % Исправляем вертикальный отступ перед заголовком оглавления
    \renewcommand{\cftloftitlefont}{\hdngaligni\Large\bfseries} % Исправляем размер заголовка списка рисунков
    \setlength{\cftbeforeloftitleskip}{-1.4\curtextsize}        % Исправляем вертикальный отступ перед заголовком списка рисунков
    \renewcommand{\cftlottitlefont}{\hdngaligni\Large\bfseries} % Исправляем размер заголовка списка таблиц 
    \setlength{\cftbeforelottitleskip}{-1.4\curtextsize}        % Исправляем вертикальный отступ перед заголовком списка таблиц
    \sectionformat{\chapter}{% Параметры заголовков разделов в тексте
        format=\hdngalign\Large\bfseries, % Исправляем размер заголовка
        top-=0.4em,                       % Исправляем вертикальный отступ перед заголовком
    }
    \sectionformat{\section}{% Параметры заголовков подразделов в тексте
        format=\hdngalign\large\bfseries, % Исправляем размер заголовка
    }
}

\ifthenelse{\equal{\theheadingsize}{1}\AND \curtextsize < \bigtextsize}{% Пропорциональные заголовки и базовый шрифт 14 пт
    \sectionformat{\chapter}{% Параметры заголовков разделов в тексте
        top-=0.2em, % Исправляем вертикальный отступ перед заголовком
    }
}

%%% Счётчики %%%

%% Упрощённые настройки шаблона диссертации: нумерация формул, таблиц, рисунков
\ifthenelse{\equal{\thecontnumeq}{1}}{%
    \counterwithout{equation}{chapter} % Убираем связанность номера формулы с номером главы/раздела
}
\ifthenelse{\equal{\thecontnumfig}{1}}{%
    \counterwithout{figure}{chapter}   % Убираем связанность номера рисунка с номером главы/раздела
}
\ifthenelse{\equal{\thecontnumtab}{1}}{%
    \counterwithout{table}{chapter}    % Убираем связанность номера таблицы с номером главы/раздела
}


%%http://www.linux.org.ru/forum/general/6993203#comment-6994589 (используется totcount)
\makeatletter
\def\formbytotal#1#2#3#4#5{%
    \newcount\@c
    \@c\totvalue{#1}\relax
    \newcount\@last
    \newcount\@pnul
    \@last\@c\relax
    \divide\@last 10
    \@pnul\@last\relax
    \divide\@pnul 10
    \multiply\@pnul-10
    \advance\@pnul\@last
    \multiply\@last-10
    \advance\@last\@c
    \total{#1}~#2%
    \ifnum\@pnul=1#5\else%
    \ifcase\@last#5\or#3\or#4\or#4\or#4\else#5\fi
    \fi
}
\makeatother

\AtBeginDocument{
%% регистрируем счётчики в системе totcounter
    \regtotcounter{totalcount@figure}
    \regtotcounter{totalcount@table}       % Если иным способом поставить в преамбуле то ошибка в числе таблиц
    \regtotcounter{TotPages}               % Если иным способом поставить в преамбуле то ошибка в числе страниц
}

\DeclareUnicodeCharacter{0306}{\todo{поправить й}}
\hfuzz=20.002pt 